%%
%% This is file `tikzposterBlockstyles.tex',
%% generated with the docstrip utility.
%%
%% The original source files were:
%%
%% tikzposter.dtx  (with options: `tikzposterBlockstyles.tex')
%% 
%% This is a generated file.
%% 
%% Copyright (C) 2014 by Pascal Richter, Elena Botoeva, Richard Barnard, and Dirk Surmann
%% 
%% This file may be distributed and/or modified under the
%% conditions of the LaTeX Project Public License, either
%% version 2.1 of this license or (at your option) any later
%% version. The latest version of this license is in:
%% 
%% http://www.latex-project.org/lppl.txt
%% 
%% and version 2.1 or later is part of all distributions of
%% LaTeX version 2014/10/15 or later.
%% 

% Options:
%   titlewidthscale
%   bodywidthscale
%   titlecenter, titleleft, titleright
%   titleoffsetx
%   titleoffsety
%   bodyoffsetx
%   bodyoffsety
%   bodyverticalshift
%   roundedcorners
%   linewidth
%   titleinnersep
%   bodyinnersep

% Parameter:
%   \ifBlockHasTitle  -  boolean
%   blocktitle  -  coordinate
%   blockbody  -  coordinate
%   \blockroundedcorners  -  number
%   \blocklinewidth  -  length
%   \blockbodyinnersep  -  length
%   \blocktitleinnersep  -  length
%   blockbodybgcolor  -  color
%   blocktitlebgcolor  -  color
%   framecolor  -  color

\defineblockstyle{Default}{
	titlewidthscale=1, 
	bodywidthscale=1, 
	titlecenter,
	titleoffsetx=0pt, 
	titleoffsety=0pt, 
	bodyoffsetx=0pt, 
	bodyoffsety=0pt,
	bodyverticalshift=-8pt, 
	roundedcorners=15, 
	linewidth=0.2cm,
	titleinnersep=1cm,
	bodyinnersep=0.65cm
}{
	\begin{scope}[line width=\blocklinewidth, rounded corners=\blockroundedcorners]
		\ifBlockHasTitle %
			\draw[color=blocktitlebgcolor, fill=blocktitlebgcolor] (blockbody.south west) rectangle (blocktitle.north east);
			\draw[color=blocktitlebgcolor, fill=blockbodybgcolor] (blockbody.south west) rectangle (blockbody.north east);
		\else
			\draw[color=blocktitlebgcolor, fill=blockbodybgcolor] (blockbody.south west) rectangle (blockbody.north east);
		\fi
	\end{scope}
}

% \defineblockstyle{Default}{
% 	titlewidthscale=1, bodywidthscale=1, titlecenter,
% 	titleoffsetx=0pt, titleoffsety=0pt, bodyoffsetx=0pt, bodyoffsety=0pt,
% 	bodyverticalshift=0pt, roundedcorners=30, linewidth=0.4cm,
% 	titleinnersep=1cm, bodyinnersep=1cm
% }{
% 	\begin{scope}[line width=\blocklinewidth, rounded corners=\blockroundedcorners]
% 		\ifBlockHasTitle %
% 			\draw[color=blocktitlebgcolor, fill=blocktitlebgcolor] (blockbody.south west) rectangle (blocktitle.north east);
% 			\draw[color=blocktitlebgcolor, fill=blockbodybgcolor] (blockbody.south west) rectangle (blockbody.north east);
% 		\else
% 			\draw[color=blocktitlebgcolor, fill=blockbodybgcolor] (blockbody.south west) rectangle (blockbody.north east);
% 		\fi
% 	\end{scope}
% }

\defineblockstyle{Basic}{
	titlewidthscale=0.8, bodywidthscale=1, titlecenter,
	titleoffsetx=0pt, titleoffsety=0pt, bodyoffsetx=0pt, bodyoffsety=15mm,
	bodyverticalshift=15mm, roundedcorners=22, linewidth=5pt,
	titleinnersep=8mm, bodyinnersep=8mm
}{
	\draw[rounded corners=\blockroundedcorners, inner sep=\blockbodyinnersep, line width=\blocklinewidth, color=framecolor, fill=blockbodybgcolor]
	(blockbody.south west) rectangle (blockbody.north east); %
	\ifBlockHasTitle%
		\draw[rounded corners=\blockroundedcorners, inner sep=\blocktitleinnersep, line width=\blocklinewidth, color=framecolor, fill=blocktitlebgcolor]
		(blocktitle.south west) rectangle (blocktitle.north east); %
	\fi%
}

\defineblockstyle{Minimal}{
	titlewidthscale=1, bodywidthscale=1, titleleft,
	titleoffsetx=0pt, titleoffsety=0pt, bodyoffsetx=0pt, bodyoffsety=0pt,
	bodyverticalshift=0pt, roundedcorners=0, linewidth=0.2cm,
	titleinnersep=1cm, bodyinnersep=1cm
}{
	\begin{scope}[line width=\blocklinewidth, rounded corners=\blockroundedcorners]
		\ifBlockHasTitle %
			\draw[draw=none]%, fill=blockbodybgcolor]
			(blockbody.south west) rectangle (blocktitle.north east);
			\draw[color=blocktitlebgcolor, loosely dashed]
			(blocktitle.south west) -- (blocktitle.south east);%
		\else
			\draw[draw=none]%, fill=blockbodybgcolor]
			(blockbody.south west) rectangle (blockbody.north east);
		\fi
	\end{scope}
}

\defineblockstyle{Envelope}{
	titlewidthscale=1, bodywidthscale=1, titlecenter,
	titleoffsetx=0pt, titleoffsety=0pt, bodyoffsetx=0pt, bodyoffsety=0pt,
	bodyverticalshift=0pt, roundedcorners=20, linewidth=1.6pt,
	titleinnersep=1cm, bodyinnersep=1cm
}{
	\begin{scope}[rounded corners=\blockroundedcorners, line width=\blocklinewidth,
			drop shadow={shadow xshift=0.3cm, shadow yshift=-0.3cm, opacity=0.3} ]
		\ifBlockHasTitle
			% the big rectangle
			\draw[color=blocktitlebgcolor, fill=blockbodybgcolor, drop shadow]
			(blockbody.south west) rectangle (blocktitle.north east);%
			\begin{scope}
				\clip (blocktitle.south west) rectangle (blocktitle.north east);
				% fading on top
				\fill[rounded corners=0, path fading=south, fill=blocktitlebgcolor, opacity=.4]
				($(blocktitle.south west)-(0.1,0)$) rectangle ($(blocktitle.north east)+(0.1,0)$);
				% the trapezium
				\draw[draw=none, bottom color=blocktitlebgcolor, top
					color=blocktitlebgcolor!85!] %
				($(blocktitle.north west)+(0.25,0)$) -- ($(blocktitle.north west)+(0.75,0)$) -- %
				($(blocktitle.south west)+(2.5,0)$) -- ($(blocktitle.south east)-(2.5,0)$) -- %
				($(blocktitle.north east)-(0.75,0)$) -- ($(blocktitle.north east)-(0.25,0)$) -- cycle;
			\end{scope}
		\else
			% No title
			\draw[color=blocktitlebgcolor, fill=blockbodybgcolor]
			(blockbody.south west) rectangle (blockbody.north east);
		\fi
	\end{scope}
}

\defineblockstyle{Corner}{
	titlewidthscale=1, bodywidthscale=1, titleleft,
	titleoffsetx=0pt, titleoffsety=0pt, bodyoffsetx=0pt, bodyoffsety=0pt,
	bodyverticalshift=0pt, roundedcorners=20, linewidth=1.2pt,
	titleinnersep=1cm, bodyinnersep=1cm
}{
	% the shadow above the corner
	\begin{scope}
		\clip (blockbody.south west) rectangle (blocktitle.north east);
		\begin{scope}[transform canvas={xshift=-1cm, yshift=-0.8cm, rotate
			around={-20:($(blocktitle.north east)-(10,0)$)}}]
			\fill[color=gray, path fading=north, opacity=0.8]%
			($(blocktitle.north east)-(10,1)$) rectangle ($(blocktitle.north east)+(2,2.3)$);
		\end{scope}
	\end{scope}
	%
	% the border
	\def \border{%
		[rounded corners=30] (blockbody.south west) -- (blocktitle.north west) %
		[rounded corners=30] -- ($(blocktitle.north east)-(9.4,0)$)
		[rounded corners=30] -- ($(blocktitle.north east)-(0,3.4)$)
		[rounded corners=30] |- (blockbody.south west) -- cycle
	}
	\draw[line width=\blocklinewidth, color=blocktitlebgcolor, fill=blockbodybgcolor,
		% drop shadow={shadow xshift=0.3cm, shadow yshift=-0.3cm, opacity=0.3}
	] \border;
	%
	% the corner
	\begin{scope}
		\def \corner{ ($(blocktitle.north east)-(0,6)$) -- ($(blocktitle.north east)-(0,4.5)$) .. %
			controls ($(blocktitle.north east)-(-0,2.7)$) and ($(blocktitle.north east)-(2.8,2.2)$)
			.. ($(blocktitle.north east)-(3.8,4.6)$) %
			.. controls ($(blocktitle.north east)-(8.6,0)$) .. ($(blocktitle.north east)-(11.4,0)$) %
			[rounded corners=30] -- ($(blocktitle.north east)-(9.4,0)$) %
			[rounded corners=30] -- ($(blocktitle.north east)-(0,3.4)$) %
			[rounded corners=0] -- ($(blocktitle.north east)-(0,6)$)}
		\draw[blocktitlebgcolor] \corner;
		\clip \corner;
		\begin{scope}[transform canvas={xshift=-1cm, yshift=-1.3cm, rotate
			around={-23:($(blocktitle.north east)-(10,0)$)}}]
			\fill[color=blocktitlebgcolor!90] ($(blocktitle.north east) - (10,2)$)
			rectangle ($(blocktitle.north east) + (2,3.6)$); %
			\fill[color=blocktitlebgcolor , path fading=south, opacity=1]
			($(blocktitle.north east) - (10,-1.2)$) rectangle ($(blocktitle.north east) + (2,1.6)$); %
			\fill[color=blocktitlebgcolor , path fading=north, opacity=1]
			($(blocktitle.north east) - (10,-1.6)$) rectangle ($(blocktitle.north east) + (2,2.1)$);
		\end{scope}
	\end{scope}%
}

\defineblockstyle{Slide}{
	titlewidthscale=1, bodywidthscale=1, titleleft,
	titleoffsetx=0pt, titleoffsety=0pt, bodyoffsetx=0pt, bodyoffsety=0pt,
	bodyverticalshift=0pt, roundedcorners=0, linewidth=0pt, titleinnersep=1cm,
	bodyinnersep=1cm
}{
	\ifBlockHasTitle%
		\draw[draw=none, left color=blocktitlebgcolor, right color=blockbodybgcolor]
		(blocktitle.south west) rectangle (blocktitle.north east);
	\fi%
	\draw[draw=none, fill=blockbodybgcolor] %
	(blockbody.north west) [rounded corners=30] -- (blockbody.south west) --
	(blockbody.south east) [rounded corners=0]-- (blockbody.north east) -- cycle;
}

\defineblockstyle{TornOut}{
	titlewidthscale=1, bodywidthscale=1, titlecenter,
	titleoffsetx=0pt, titleoffsety=0pt, bodyoffsetx=0pt, bodyoffsety=0pt,
	bodyverticalshift=-1.2cm, roundedcorners=0, linewidth=1.2pt,
	titleinnersep=1cm, bodyinnersep=1cm
}{
	\ifBlockHasTitle%
		\coordinate (topright) at (blocktitle.north east);
	\else
		\coordinate (topright) at (blockbody.north east);
	\fi%
	\draw[color=blocktitlebgcolor, fill=blockbodybgcolor,%
		line width=\blocklinewidth, drop shadow={shadow xshift=0.2cm, shadow yshift=-0.2cm,opacity=0.3}, %
		decorate, decoration={random steps,segment length=1.5cm,amplitude=0.15cm}
	] (blockbody.south west) rectangle (topright);%
}



\endinput
%%
%% End of file `tikzposterBlockstyles.tex'.